\documentclass{article} 
\begin{document}

\newline

Today, our greatest fear is Silence. Less for the ears, more for the
soul. We bathe in entropy and Silence threatens to pull the plugs of our
warm bathtubs. Why is this entropy so comforting? \textit{Maybe} our
neurons have short-circuited because of certain biological incentives;
OUR BIOLOGY NEEDN'T TRANSCEND STIMULI. \textit{Maybe} this is
a statistical - physics phenomenon, an emergent phenomenon, the way
daylight blue is an emergent phenomenon of light scattering off Earth's
molecular sky; maximizing and deifying entropy might be our collective
obedience to Physical Law; OUR LONGING FOR FREE WILL NEEDN'T TRANSCEND
THE LAWS OF PHYSICS. \textit{Maybe} it's the Human Condition: the
entropy comforts us because we are incapable of comforting ourselves;
when we try, our minds recede into infinities like the boundaries of our
visible Universe; OUR INTERNAL RECURSIONS NEEDN'T TRANSCEND THE BEHEMOTH
OF METAPHYSICAL INFINITY. \textit{Maybe} the perception of Silence is an
artifact of Language; it could turn out that what we call Silence is
grossly meaningless and ill-posed, perhaps as meaningless as the
question ``Is Love Orange? `` So it might turn out that it isn't Silence
that we're afraid of, so much as it's the proposition of meaninglessness
that it (Silence) draws our attention to. After all, is there such
a thing as Ontological Silence? One always hears one's own heart-beat,
and feels synaptic firing of some sort; to hear and feel nothing is to
be, literally, un-alive. OUR LANGUAGE NEEDN'T TRANSCEND JACK SHIT.
\textit{Maybe} all this is a failure in discerning: How can you tell,
with certainty, that a person is simply being quiet, i.e. ruminative,
patient, sensitive, even morally generous versus being abyssal, and
roaring with internal existential Silence, the kind of Silence with the
reputation of being pulse-rate-jackingly but also \textit{unknowably}
scary. OUR INFERENCE NEEDN'T TRANSCEND OUR IGNORANCE. \textit{Maybe} the
factors are economic: Capitalism relies on profit, which relies on
commodification, which relies on growth of Stuff and Services, and so it
is conceivable that acquiring and contributing to S\&S gives us all the
purpose and narrative foundation for The Good Life, until we realize,
harshly (is this adverb's sole purpose to qualify
\textit{realization}?), that acquiring and contributing to S\&S is
boring, unfulfilling, and requires a medication of the soul, which
medication comes only in the form of, surprise surprise, more S\&
Mother-Lovin' S, the cycle of which is - I don't need to Euclidesquely
prove to you - the terminal ass-kicking of Silence, that great third but
deeply incompatible S. OUR ECONOMICS NEEDN'T TRANSCEND OUR ENNUI.
\textit{Maybe} Silence isn't even the fear in the first place, it is the
\textit{effect} of the aforementioned causes: Biological imprisonment,
Physical limitation, Metaphysical loneliness, Symbolic inadequacy,
Interpersonal impenetrableness, Economic impermissibleness.
Respectively. All of which - it can be argued and proved with sufficient
grant money - are inextricably related. Are you with me so far? Did you
get all of that? Do you have anything to say? Your eye-lids betray your
boredom. \newline\newline


Now get out of my office. And let Susan in. \newline\newline

Today, our greatest fear is the Lack of Freedom...

\end{document}
