\documentclass{article} 
\begin{document}

\newline

``I cannot love you like I love myself. I can only love you the way
I wish to be loved.''
\newline
\newline

``Are you saying your love for yourself is different from the way you
wish to love yourself?''
\newline
\newline

``I'm saying how I love myself is fundamentally different from how
I wish to be loved.''
\newline
\newline

``Why can't they be one and the same?'' 
\newline
\newline

``Your question's logical limit is the ridiculous proposition:
\textit{Why can't two persons be the same?} They cannot. Even if they
are the absolute spitting reflections of each other, in mind and body,
cosmically cloned twins, they are two persons occupying two different
spatial locations. Demarcated, thus identifiable. There is a boundary
between each one's self and the other. They can be two and the same, but
they cannot be \textit{one and the same}.'' \newline \newline

``So can your love for yourself and the love you wish you were given be
\textit{two and the same}?'' \newline
\newline

``No they can't. Examine your words. One is an ability, the other
a wish. One is my love for myself. The other is a wish for a kind of
love that I cannot attain by virtue of being \textit{inside} myself.
I cannot \textit{see} all of myself from \textit{inside}. Only parts of
it. And that is the extent of my self-love. In verse:\newline\newline

\textit{In my temple I kneel and pray, enclosed, alone}\newline

\textit{Standing Outside, you behold a glorious monument of
stone}\newline 

\textit{And even if through these heaven-supporting pillars you peer}
\newline 

\textit{Deaf to the prayer, only the praying is clear} \newline\newline

The unknown part of myself, that part which I cannot access by virtue of
being \textit{inside} myself, becomes impossible to love. It would be
a love of vacuum, a vacuum of love, i.e. no love. It can
only be done from \textit{outside}.'' \newline
\newline

``So the love you wish is really a wish to be \textit{seen} a certain
way?'' \newline
\newline

``If you look at it that way, this love is no different from loving
myself. I would be using my lover as a mere agency for my own
self-loving. This is what it means to love being \textit{seen} a certain
way. The love \textit{I} wish however is that which comes
\textit{despite} being seen a certain way.'' \newline
\newline

``So you wish to be loved unconditionally then.'' \newline
\newline

``No. I wish to be loved very conditionally. I wish to be loved for who
I am and not loved for loving's sake. The love I wish is that which
permits me to be myself. ''\newline
\newline

``But if you wish to be loved for who you are, how can you be sure that
who you are merits love?'' \newline
\newline

``Indeed. I don't. I never said I \textit{deserved} love. I'm only
expressing a desire for it.''\newline
\newline

``Doesn't this get complicated by the reality of human nature? Don't you
need to love your lover in return? In fact, it is the value of your
returned love that produces your lover's love for you.''\newline
\newline

``The reason I love my lover is not because he loves me, but because he
is the realization of my ideals. My bent-into-the-shape-of-reality
ideals. Imperfect, both him and my love for him. Do I love him because
he loves me? That might be the cause, but not the reason.'' \newline
\newline

``You're deep.''\newline
\newline

``You're sheep.''\newline
\newline

``You're funny.''\newline
\newline

``You most certainly are not.''\newline
\newline

``You like to have the last word.'' \newline
\newline

``I like my words to last.'' \newline
\newline

``You like word play.''\newline
\newline

``I like word sex.''\newline
\newline

``And weird sex.''\newline
\newline

``Can we go on and on making love?''\newline
\newline

``Yes we can.''\newline
\newline

\end{document}
