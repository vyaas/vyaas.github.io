\documentclass{article} 
\begin{document}

\newline

One day your fear will be realized. When the phone rings, your body
trembling with a cardiac dub-dub as you press green and hear that
a Loved One has died, your fear will be realized. You will squeeze the
talking brick, activate needless features like camera, volume control,
and God forbid google-voice wondering if buttons placed on the sides of
phones were precisely intended for times like this, Man juxtaposing
Life's absurdities with Technology's. You will, with a mood and maturity
that surprises you, understand that the speaker is as much on the verge
of self-destruction as you are. That speaking of death is as unbearable
as hearing about it. That this is the precise connotation of the word
\textit{Death} will dawn on you like it never has before. And in this
exchange of electromagnetically ricocheted utterances, these
everything-that-has-to-be-said-has-been-saids, you swell with thoughts
of profundity. Because you are so far away, geographically and thus
emotionally (yes, to you who especially appreciates that one's
surrounding \textit{stirs} one's being (which is why you moved out here
in the first place), you will concede that being absent from your L.O.'s
surrounding doesn't facilitate a whole lot of mutual stirring, and that
this implies emotional distance, and whether you know it or not, the
word ``emotion'' comes from the Latin \textit{motum} which surprise
surprise means \textit{to stir}.), you can't help but assemble
abstraction upon abstraction, not unlike a theorem far from its axioms,
about the meaning of this Death, the Life swerving implications of this
Death, the what-does-it-mean-to-be-alive didactics of this Death. Of
course, it is natural to feel these things, but except that to you these
are less like feelings and more like contemplations, the realization of
which makes you drop your phone. Because you \textit{want to feel}. Your
desire for feeling is not unlike that of a madman's in the desert,
whose thirst is itself a mirage to him. You pick up your phone, hastily
apologize into it, which looks like an apology \textit{to} it because
upon saying ``I'm coming.'', you hurl your phone at the nearest wall,
like this country's best baseball pitcher, and you watch your phone
break into glorious assembly-line-reminiscent pieces. You delight in
this new entropy. You scroll through possible plane tickets but only
after spending a minute of silence worrying about how to inform your
stirrers (and stirrees?) in proximity here; how will they emotionally
nurture you when they don't know the circumstances leading to your
sudden departure, let alone know how emotionally starved you are. And so
you pick a flight, for a paycheck burning \$1800 USD because you are
keenly aware that any layover time will all but make you fall off the
cliffs of your sanity. And when you read the words ``Congratulations!
Your Credit Card has been approved'', you feel a resurgence of appetite
for entropy. But your laptop is your last \textit{convenient} means of
communication so you bring your knees to your chest and hug yourself in
a pathetic attempt to squeeze shut your entropic appetite. You stuff
a t-shirt and a pair of boxers into a backpack and are appalled by
hygiene's say in anything. Your phone is silicon cereal so calling
a cab will have to be done unconventionally through your laptop, but
this time you're better prepared to receive a Congratulations. Except
you aren't because you've just started worrying about your inability to
be moved to tears at a time like this. And you're actually not even
really worried about this lack of natural movement to tears but about
how you know \textit{precisely how to produce tears not just now but
whenever you feel like}. You simply close your eyes, and conjure images
of the L.O. in his/her most extraordinary plainness. You see him/her
laugh, and dwell on the innocence of it. You see him/her look at the
world around, whether in awe at the spectral sunset over hills, or in
dread at the gathering grey of clouds, or in indifference at a bug
he/she just squished. And most tear-triggeringly of all, you see him/her
do whatever it is he/she is devoted to, and dwell on how \textit{lost}
he/she is in it. And this being \textit{lost} to you is the most
relatable and moving thing; here by \textit{lost} you mean ``removed
from all else''. You know that this precisely is what comes closest to
making you feel sad, if you're capable of feeling anything at all.
Because your inaccessibility into the living space, the \textit{being
space}, of an individual occupied with what clearly appears to you as
being his/her \textit{raison}, his/her ritualistic exercise to keep
boredom at bay (which who said ritual means unthinking? It is really the
end point of thinking, it is \textit{revelation}.), equals your own
loneliness. That if loneliness means anything, it is this, that everyone
feels it \textit{in relation to everybody else}. And now your mentation
includes the real un-metaphysical i.e. totally no-fucking-around
physical \textit{loss} of this L.O., this \textit{Lonely Other}, making
for a doubly dense pang of sadness, which finally induces you to cry,
which self-inducements you never once doubted because they've never
failed even during your happiest moments (tears of ``joy'' indeed).
\textit{Eyeagra falls}. And so you're crying and you mouse over your
address on a javascript rendered map, type in the convenient
abbreviation \textit{LAX}, wait for a javascript rendered pinwheel or
hourglass designed to sustain an illusion of continuum to complete its
animation, view a blurry Congratulations because your eyes are just too
damn sad-sogged and the internet, although not neutral, is vastly
indifferent. You sling your bag on, tie your shoes, turn all that you
can off, regret that you should have really done the dishes yesterday
when you had the time but were instead binge watching in one of many of
today's culturally acceptable ways of pissing away one's life, lock the
door, step into an unbearably indifferent outside, get into the car,
desperately avoid any talk, even the smallest of small talk if you can
and this means holding back your blubbering and looking out the window
and away from the driver to avoid making red eye contact with him (slash
her, who knows?). You'd ponder about quantum teleportation if you had
the relaxing space and circumstance which you so totally are not in
possession of now. And now it occurs to you that you will have to sit in
two different airplanes, one for fourteen hours and the other for five
hours, surrounded by other raison fulfilling beings with whom any kind
of speech act ought to be avoided on all possible accounts, and if you
do begin to cry and are unlucky not to have an aisle seat and so are
stuck and this makes you cry a little more and an air-hostess looks at
you and offers you something to drink as though rehydration is the
answer (although in fairness what \textit{else can she do}?), you accept
the drink because God knows that's the only way the Universe gives
a shit if it gives one at all; not crying would not elicit any Universal
Compassion whatsoever and would leave you morbidly alone. And then you
will land not once but \textit{twice}, making the whole symbolically
instructive act of coming down to Earth from your cancerous mentations
lose whatever the fuck narrative charm it had. One day, all this will
happen and your fear will be realized. Don't say I didn't warn you.


\end{document}
