\documentclass{article} 
\begin{document}

\newline

There once was a boy who pilfered money from his mother's purse. When
his mother found out he'd been the culprit (and not the house maid) she
slapped him at once (but never apologized to the maid for suspecting
her). And when he told his mother that he'd been stealing money to buy
comic books, she slapped him again, only harder this time, with a loud
and narratively fitting ``Pow!'', her violent response an example of the
kind of attachment to values that typifies older generations. The boy
was too immature to grasp the parallel between his mother's
backhand-forehand slapping and the kind of ass-kicking his comic book
superheros were more than inclined to serve to the morally crooked,
which included stealers of all kinds. It was precisely this kind of
immaturity that the mother had had in mind when she refused to entertain
the boy's repeated demands for comic books, reasoning that the boy was
too young and foolish to resolve any kind of morally spectral literature
for himself; this denial is why he ended up stealing the money in the
first place. What the mother was unaware of, however, was that while
beating the fear of God into her son, she beat his inner child out of
him: the boy would grow up to be colder than the chilled beer he would
come to rely on for everything between casual company to maximally
requited Love. The mother also didn't realize that the morality she
thought she was instilling in the boy by beating him was actually being
effortlessly cancelled out by the boy's not so immature observation that
the mother was being totally hypocritical in her spiel about
righteousness by failing to apologize to the maid for having mistakenly
suspected her of stealing and, worse, making similar groundless
allegations in the future while cognizant of the fact that the maid
could not simply quit because this job paid the medical bills which
resulted from her daily drunk husband beating her for her money for some
daily drinking. What makes the mother's hypocrisy especially pernicious
is that she used the repeated allegations as a device to instil fear and
self-doubt in the maid, managing to keep her self-esteem and thus wages
low. All this the boy knew, but would not help the maid, even when he
was qualified to vote, because moral reasoning in the house was pretty
much self-serving and fear-driven, he reasoned that because this was
her only source of income he shouldn't interfere and possibly make her
situation worse, and besides, the poor possess the secret to happiness
or else why the hell are they all not committing suicide in the
millions? And so the boy was complicit in the grotesque unfairness and
thus had more reason to pass out daily in various street corners in his
later realization filled years.

\newline\newline

There is a point in this story about Morals, but the Moral of the story
is to not point it out.

\end{document}
