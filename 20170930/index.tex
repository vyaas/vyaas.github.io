\documentclass{article} 
\begin{document}

\newline

A:...

B: Social lubricant. Plain and simple.
\newline
\newline

A:...

B: I wouldn't say they so much \textit{slip} out of my mouth as they
\textit{glide} out of it, at least like before the fourth drink. After
that, yeah, I'll admit to some significant slippage. \newline
\newline
\newline

A:...

B: Speaking to the speech-capable is psychically tense for me. Being
a congenital liar myself, I'm unable to tell if I'm ever being taken
seriously. Which is weird because in order to be a good liar, one ought
to like \textit{gauge} the listener, like stay psychically abreast of
their mentation, by paying attention to the rhythm of his - the
listener's - deliberate and default words with the diaghragm of your
like psychic receiver, and by seeking a pattern in the reticulated mess
of insecurities/principles revealed in his countenance, and by mapping
his bodily movements and stases with these very pickled
insecurities/principles he's preserving in his skull for God knows what,
etc., all of which I congenitally do, except that it additionally occurs
to me that the listener - he - could also be engaged in surveying
\textit{my} mentation, and that he now, sub-optically of course,
\textit{sees} me seeing him seeing me and so on. Then, it is a matter of
psychic tennis you see, one of us has to commit a fault for it to end.
Or if you prefer, one of us has to truncate this like infinite series
expansion. \newline \newline

A:...

B: That's right. The more I drink, the sooner I commit the fault, or the
fewer terms in the series if you will. Its a relief really. The kind
you experience when you take an overdue shit.
\newline
\newline

A:...

B: Call it narcissism or solipsism or even alcoholism. I don't give
a philosophical fuck.
\newline
\newline

A:...

B: Yes. I love you.
\newline
\newline

\end{document}
